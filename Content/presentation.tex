%!TEX TS-program = xelatex

\documentclass[t]{beamer}  % [t], [c], или [b] --- вертикальное выравнивание на слайдах (верх, центр, низ)
%\documentclass[t, handout]{beamer} % Раздаточный материал (на слайдах всё сразу)
%\documentclass[aspectratio=169, t]{beamer} % Соотношение сторон

\usepackage{epstopdf}

%\usetheme{Berkeley} % Тема оформления
%\usetheme{Bergen}
%\usetheme{Szeged}

%\usecolortheme{beaver} % Цветовая схема
%\useinnertheme{circles}
%\useinnertheme{rectangles}

\usepackage{HSE-theme/beamerthemeHSE} % Подгружаем тему


%%% Работа с русским языком
\usepackage{cmap}					% поиск в PDF
\usepackage{mathtext} 				% русские буквы в формулах
\usepackage[T2A]{fontenc}			% кодировка
\usepackage[utf8]{inputenc}			% кодировка исходного текста
%%% Работа с русским языком и шрифтами
\usepackage[english,russian]{babel}   % загружает пакет многоязыковой вёрстки


%%% Дополнительная работа с математикой
\usepackage{amsmath,amsfonts,amssymb,amsthm,mathtools} % AMS
\usepackage{icomma} % "Умная" запятая: $0,2$ --- число, $0, 2$ --- перечисление

%% Номера формул
%\mathtoolsset{showonlyrefs=true} % Показывать номера только у тех формул, на которые есть \eqref{} в тексте.
%\usepackage{leqno} % Нумерация формул слева

%% Свои команды
\DeclareMathOperator{\sgn}{\mathop{sgn}}

%% Перенос знаков в формулах (по Львовскому)
\newcommand*{\hm}[1]{#1\nobreak\discretionary{}
	{\hbox{$\mathsurround=0pt #1$}}{}}

%%% Работа с картинками
\usepackage{graphicx}  % Для вставки рисунков
\graphicspath{{images/}{images2/}}  % папки с картинками
\setlength\fboxsep{3pt} % Отступ рамки \fbox{} от рисунка
\setlength\fboxrule{1pt} % Толщина линий рамки \fbox{}
\usepackage{wrapfig} % Обтекание рисунков текстом
\usepackage{caption}


%%% Работа с таблицами
\usepackage{array,tabularx,tabulary,booktabs} % Дополнительная работа с таблицами
\usepackage{longtable}  % Длинные таблицы
\usepackage{multirow} % Слияние строк в таблице

%%% Программирование
\usepackage{etoolbox} % логические операторы

%%% Другие пакеты
\usepackage{lastpage} % Узнать, сколько всего страниц в документе.
\usepackage{soul} % Модификаторы начертания
\usepackage{csquotes} % Еще инструменты для ссылок
%\usepackage[style=authoryear,maxcitenames=2,backend=biber,sorting=nty]{biblatex}
\usepackage{multicol} % Несколько колонок

%%% Картинки
\usepackage{tikz} % Работа с графикой
\usepackage{pgfplots}
\usepackage{pgfplotstable}
\usepackage{verbatim}
\usetikzlibrary{fadings}
\usepackage[outline]{contour}

\usepackage{chngcntr} % нумерация графиков и таблиц по секциям
\counterwithin{table}{section}
\counterwithin{figure}{section}

\input{Parts/rustcode.tex}
\usepackage{setspace}
\usepackage{color}


\title{Реализация поддержки асинхронного программирования для фреймворка \texttt{DSLab}}
\author{Артём Макогон}
\date{\today}
% \institute[Высшая школа экономики]{National Research University\\ 
	% <<Higher School of Economics>>}



\begin{document}
	
	\begin{frame}
		\maketitle
	\end{frame}
	

    \section{Введение}
    \subsection{Описание предметной области}


    \begin{frame}
    \frametitle{\insertsection} 
	\framesubtitle{\insertsubsection}

	\begin{figure}
		\centering
		\includegraphics<1>[width=\linewidth]{images/event_pipeline_0}
		\includegraphics<2>[width=\linewidth]{images/event_pipeline_1}
		\includegraphics<3>[width=\linewidth]{images/event_pipeline_2}
		\includegraphics<4>[width=\linewidth]{images/event_pipeline_3}
		\includegraphics<5>[width=\linewidth]{images/event_pipeline_4}
		\includegraphics<6>[width=\linewidth]{images/event_pipeline_5}
		\includegraphics<7>[width=\linewidth]{images/event_pipeline_6}
		\includegraphics<8>[width=\linewidth]{images/event_pipeline_7}
		\includegraphics<9>[width=\linewidth]{images/event_pipeline_8}
		\includegraphics<10>[width=\linewidth]{images/event_pipeline_9}
		\label{simulation}
		\caption*{Исполнение симуляции}
	\end{figure}
        
    \end{frame}

	\subsection{Архитектура проекта DSLab}
	\begin{frame}
		\frametitle{\insertsection} 
		\framesubtitle{\insertsubsection}

		\begin{figure}[H]
			\centering
			\includegraphics[width=0.9\linewidth]{images/dslab_arc}
			\caption*{Архитектура DSLab}
			\label{dslab_arc}
		\end{figure}
	\end{frame}


	\subsection{\texttt{Callback} модель. Реализация пользовательского Process.}
	\begin{frame}[fragile]
		\frametitle{\insertsection} 
		\framesubtitle{\insertsubsection}
		\begin{columns}[c] 
			\begin{column}{0.8\textwidth} % First column
		\begin{figure}
			\footnotesize
			\centering
			\begin{rustcode}[escapeinside={(*@}{@*)}]
fn on(&mut self, event: Event) {
  cast!(match event.data {
    Start {} => {
      self.on_start(); (*@\textcolor{blue}{\rule{1em}{1em}}@*)
    }
    DownloadCompleted {data} => {
      self.on_download_completed(data); (*@\textcolor{red}{\rule{1em}{1em}}@*)
    }
    DiskWriteCompleted => {
      self.on_disk_write_completed();(*@\textcolor{green}{\rule{1em}{1em}}@*)
    }
  })
}
			\end{rustcode}
			\caption*{Реагирование на события.}
		\end{figure}

	\end{column}
	
	\begin{column}{0.24\textwidth}
		\begin{figure}
			\scriptsize
			\centering
			\begin{rustcode}[escapeinside={(*@}{@*)}]
impl Process{
    (*@\textcolor{red}{\rule{1.5em}{1.5em}}@*)

    (*@\textcolor{blue}{\rule{1.5em}{1.5em}}@*)

    (*@\textcolor{green}{\rule{1.5em}{1.5em}}@*)
}
			\end{rustcode}
		\end{figure}
	\end{column}
	\end{columns}

	\end{frame}

	\subsection{Преимущества асинхронного подхода}
	\begin{frame}[fragile]
		\frametitle{\insertsection} 
		\framesubtitle{\insertsubsection}

		\begin{figure}
			\footnotesize
			\centering
			\begin{rustcode}
async fn add_file_to_storage(some_file) {
  send_file_to_all_replicas(some_file);
  result = wait_confirmation_from_all().await;

  if result.has_quorum {
    send_commit_to_replicas(result.nodes);
    wait_commit_confirmation_from(result.nodes).await;

    send_ok_message_to_user();
  } else {
    send_reject_message_to_user();
  }
}
			\end{rustcode}
			\caption*{Псевдокод асинхронного взаимодействия нод в симуляции}
		\end{figure}


	\end{frame}

	\subsection{Постановка задачи}
	\begin{frame}[fragile]
		\frametitle{\insertsection} 
		\framesubtitle{\insertsubsection}

		
		Целью является реализация поддержки асинхронного программирования для фреймворка \texttt{DSLab}. Для этого необходимо:

		\begin{itemize}
			\item Реализовать асинхронное расширение для существующего ядра dslab-core.
			\item Добавить примеры использования нового функционала высокоуровневыми компонентами.
			\item Написать подробную документацию нового API и покрыть реализацию тестами.
		\end{itemize}


	\end{frame}



\end{document}